\chapter{Introducción}

En este Trabajo Fin de Grado se va a diseñar, implementar y evaluar una plataforma basada en el paradigma Publicador-Suscriptor \cite{pub-sub-solutions} para un caso de uso en agricultura, utilizando el sistema \ac{ROS} \cite{ros2}. Esto es un conjunto de librerías y herramientas para creación de aplicaciones orientadas a creación de robots. En esta plataforma se identifican dispositivos muy variados divididos en dos grupos genéricos: sensores y actuadores. Además, se contará con un controlador que recopile la información y un gestor de la base de datos para almacenarlo. Por último se dispondrá de una interfaz web para la visualización de los datos más relevantes. A destacar, se utilizarán diferentes dispositivos hardware para realizar el despliegue y poder simular algunos elementos de un escenario real.

\section{Descripción del problema}

El problema viene en que los agricultores necesitan saber el estado del terreno o de la plantación durante el periodo de trabajo para obtener el máximo rendimiento posible sin pérdidas en la cosecha. Por ello en el caso de que las condiciones climatológicas cambien o se origine una plaga, tienen que responder a estas acciones de una forma u otra.

En algunos casos las propias máquinas ya lo tienen automatizado, pero hasta donde el autor conoce muy pocos tienen una interfaz o un tipo de respuesta que les confirme que este trabajo ha sido realizado y que los trabajadores puedan consultarlo desde sus casas. Por lo tanto la mayoría de ellos acaban acudiendo al terreno a comprobar este correcto funcionamiento.

\section{Motivación}

Esta temática de trabajo ha sido seleccionada ya que el sector de la tecnología aplicada en la agricultura está en auge. Han surgido empresas con productos muy variados estos últimos años, ofreciendo diferentes servicios según el interés de los clientes.

Como el crecimiento y las novedades en la tecnología aumentan de manera exponencial, algunos de estas instalaciones pueden quedarse obsoletas en poco tiempo según las herramientas utilizadas.

Por ello se ha escogido este tema, para proponer un proyecto el cual utiliza herramientas modernas para transmitir, automatizar y almacenar una serie de datos de manera digital, gracias a las \ac{TICs}.

\section{Objetivos}

Vamos a destacar en principio tres objetivos a cumplir, y tras esto una breve explicación de ellos.

En principio, se va a realizar un diseño que permita monitorizar y automatizar procedimientos en una explotación agrícola.

El siguiente es la posibilidad añadir nuevos tipos de sensores o actuadores, sin afectar poco o nada a los demás componentes de este diseño. Un ejemplo sería un robot para transporte o un dron que realice un recorrido mientras vierte un producto.

Por último almacenar estos datos, para luego poder tratarlos y procesarlos para su visualización. Por destacar, en un trabajo futuro podrían ser utilizados para el entrenamiento de alguna \ac{IA}.

\section{Estructura del documento}

En el capítulo 2 se va a abordar el estado del sector a tratar, las tecnologías aplicadas por las diferentes empresas y un listado de las diferentes herramientas posibles a utilizar en el proyecto.

En el capítulo 3 se habla sobre la metodología aplicada, como ha sido llevado a cabo el proyecto con su respectivo diagrama de Gantt y un esquema del presupuesto total.

El capítulo 4 analiza el proyecto, que es necesario desarrollar y en qué partes se va a dividir.

En el capítulo 5 se explica la implementación realizada del proyecto, siguiendo el orden y estructura establecida en la parte del análisis.

El capítulo 6 presenta diferentes escenarios evaluados, describiendo que componentes se utilizan y midiendo la memoria al ejecutar ciertos procesos.

Por último, en el capítulo 7 se hace una reflexión sobre todo lo aprendido y proyectado en este trabajo, obteniendo conclusiones sobre ello. Tras esto se finaliza destacando algunas tareas a realizar en el futuro.