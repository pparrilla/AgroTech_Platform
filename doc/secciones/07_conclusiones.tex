\chapter{Conclusiones y trabajo futuro}

En este trabajo se han conseguido cumplir los tres objetivos principales propuestos en la introducción. Entre ellos destacar la posibilidad de añadir diferentes nodos al sistema sin tener que realizar modificaciones o solo las mínima. Esto ha sido posible gracias al haber hecho uso de ROS 2 junto al paradigma Publicador-Suscriptor, únicamente teniendo que definir un publicador o suscriptor nuevo relacionado con los tópicos existentes en esta red.

Tras el estudio de algunas de las herramientas ofrecidas por ROS 2, se han analizado todas las funcionalidades que nos otorga. Además se han aprendido diferentes conceptos relacionados con esta tecnología utilizada en más escenarios de los que se pensaba al principio, y la potencia que tiene al utilizarse en el ámbito de IoT.

Respecto a la interfaz se han podido comprender las bases del funcionamiento del Angular y la estructura de un proyecto de este tipo, diferenciando componentes, módulos y derivados.

Además, gracias a las evaluaciones de los diferentes escenarios, se ha llegado a la conclusión que la Raspberry Pi3B es más que solvente para la ejecución de los nodos requeridos, sin ver influencias claras respecto a los datos publicado en la red producidos por los demás nodos.

Para comentar las tareas que han quedado por realizar se pueden leer en la siguiente sección.

\section{Trabajo futuro}

Debido al uso de diferentes tecnologías de las cuales se desconocían su funcionamiento y como trabajar con ellas, tras un periodo de estudio y asimilación de conceptos se ha podido lograr el trabajo anterior. Pero hay puntos por realizar, que en el caso de algunos de ellos, si se hubiese querido implementar todas sus funcionalidades, probablemente daría para otro documento de similar extensión a este.

\begin{itemize}
    \item \textbf{Capa de Seguridad.} Es necesario encriptar la información y la gestión de los diferentes usuarios en el sistema, añadiendo las tablas que sean necesarias en la base de datos.
    \item \textbf{Extensión de la API REST.} Esta tarea tiene mucho potencial para futura interacción con el sistema, pudiendo acceder a estos datos para ser tratados con distintos fines.
    \item \textbf{Test de carga y despliegue de los dispositivos en un entorno real.} Para ello es necesario un estudio con personal con más conocimientos sobre los diferentes sensores en el mercado y que sería de necesidad en estos escenarios.
\end{itemize}
