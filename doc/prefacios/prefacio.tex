\thispagestyle{empty}

\begin{center}
{\large\bfseries Diseño e implementación de una plataforma escalable para explotaciones agrarias }\\
\end{center}
\begin{center}
Pedro Miguel Parrilla Navarro\\
\end{center}
%\vspace{0.7cm}

\vspace{0.5cm}
\noindent{\textbf{Palabras clave}: \textit{Pub-Sub, agricultura, DDS, ROS, micro-ROS, Firebase, SQLite3, FastAPI, Angular}}
\vspace{0.7cm}

\noindent{\textbf{Resumen}}\\

En el sector agrícola está avanzando a pasos agigantados gracias a la inclusión de las diferentes tecnologías de comunicación, robóticas o todo lo relacionado con IoT. Las empresas están desarrollando e implementando sus plataformas en el mercado abarcando un gran número de clientes. En el documento se van a analizar los productos disponibles de manera genérica y tras esto realizar la propuesta de uno propio.

En este proyecto se va a diseñar e implementar una plataforma que utiliza comunicación un patrón Publicador-Suscriptor utilizando el conjunto de herramientas que proporciona ROS 2. Este trabaja sobre una capa DDS donde se implementa ese tipo de conexión. Tras la definición y conexión de los diferentes dispositivos, la información generada será almacenada en una base de datos para su posterior visualización en una interfaz web. Por último se realiza una evaluación de diferentes escenarios, añadiendo elementos de manera progresiva al hardware disponible.

\cleardoublepage

\begin{center}
	{\large\bfseries Design and implementation of a scalable platform for agricultural farms}\\
\end{center}
\begin{center}
	Pedro Miguel Parrilla Navarro\\
\end{center}
\vspace{0.5cm}
\noindent{\textbf{Keywords}: \textit{Pub-Sub, agriculture, DDS, ROS, micro-ROS, Firebase, SQLite3, FastAPI, Angular}}
\vspace{0.7cm}

\noindent{\textbf{Abstract}}\\
The agricultural sector is advancing very fast due to the use of different communication technologies, robotics and everything related to IoT. Companies are developing and implementing their platforms in the market covering a large number of customers. In this document we are going to analyse the available products in a generic way and after this we are going to make a proposal for one of our own.

In this project we are going to design and implement a platform that uses a Publisher-Subscriber communication pattern using the set of tools provided by ROS 2. This works on a DDS layer where this type of connection is implemented. After the definition and connection of the different devices, the information generated will be stored in a database for later visualisation in a web interface. Finally, an evaluation of different scenarios is carried out, adding elements progressively to the available hardware.

\cleardoublepage

\thispagestyle{empty}

\noindent\rule[-1ex]{\textwidth}{2pt}\\[4.5ex]

D. \textbf{Juan Manuel López Soler}, Profesor del Departamento de la Teoría de la Señal, Telemática y Comunicaciones, y D. \textbf{Juan José Ramos Muñoz}, Profesor del Departamento de la Teoría de la Señal, Telemática y Comunicaciones

\vspace{0.5cm}

\textbf{Informo:}

\vspace{0.5cm}

Que el presente trabajo, titulado \textit{\textbf{Diseño e implementación de una plataforma escalable para explotaciones agrarias}},
ha sido realizado bajo nuestra supervisión por \textbf{Pedro Miguel Parrilla Navarro}, y autorizo la defensa de dicho trabajo ante el tribunal
que corresponda.

\vspace{0.5cm}

Y para que conste, expiden y firman el presente informe en Granada a Junio de 2018.

\vspace{1cm}

\textbf{El/la director(a)/es: }

\vspace{5cm}

\noindent \textbf{Juan Manuel López Soler} \hspace{3.5cm} \textbf{Juan José Ramos Muñoz}

\chapter*{Agradecimientos}

Gracias a todos aquellos profesores que me han marcado y ayudado durante mi etapa de estudiante.

Gracias a mis amigos y compañeros de piso quienes me han acompañado durante esta etapa universitaria, tanto en las buenas, como en las malas.

Gracias a Alex por estar siempre ahí cuando necesito que me echen una mano.

Gracias a MJ por ayudarme y cuidarme durante estos meses de trabajo.

Gracias a mis padres y a mi hermano por haberme apoyado, orientado y hasta regañado en los momentos necesarios.

Y por último volver a destacar las gracias a mi padre, ya que sin él seguramente este proyecto no habría sido abordado de la misma manera.